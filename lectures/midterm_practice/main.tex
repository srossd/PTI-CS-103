\documentclass[answers,addpoints]{exam} %answers,addpoints]{exam}

\usepackage{style}

\title{
    \vspace{-1cm}
	\horrule{1pt}\\[.5em]
	\huge Practice Midterm Exam
	\horrule{2pt}
	\vspace{-1cm}
}
\author{}
\date{}

\begin{document}

\maketitle
    
\begin{center}
    \fbox{\fbox{\parbox{4.5in}{\centering Answer the questions in the spaces provided on the question sheets. If you run out of room for an answer, continue on the back of the page.}}}
\end{center}

\vspace{0.2in}
\makebox[\linewidth]{Name:\enspace\hrulefill } \\[2em]
\makebox[\linewidth]{Date:\enspace\hrulefill }

\vspace{1in}
\begin{center}
    \gradetable[h][pages]
\end{center}

\pagebreak

\section{Multiple Choice}

\begin{questions}

\question[2] Which of the following is not an example of computer software?
\begin{choices}
    \CorrectChoice Memory storage
    \choice Internet Explorer
    \choice Word processors
    \choice Video games
\end{choices}

\question[2] Which of the following lines of codes is commented with the correct syntax? 
\begin{choices}
    \choice ** System.out.print("Hello"); **
    \choice ** System.out.print("Hello");
    \CorrectChoice // System.out.print("Hello");
    \choice /\& System.out.print("Hello");
\end{choices}

\question[2] Which of the following is a correct definition of the Central Processing Unit?
\begin{choices}
    \CorrectChoice hardware for performing instructions
    \choice hardware for storing and recalling data
    \choice hardware allowing the computer to interact with the user directly
    \choice hardware for processing keystrokes from the keyboard
\end{choices}

\question[2] Which of the following could fill in the blank in the following line of code:
    \begin{center}
        \ic{String a = ??????}
    \end{center}
\begin{choices}
    \choice \ic{5}
    \CorrectChoice \ic{"5"}
    \choice \ic{'5'}
    \choice \ic{'five'}
\end{choices}

\question[2] What does the following code segment print out?
\begin{code}
    System.out.print(5/2.0);
\end{code}
\begin{choices}
    \choice 2
    \choice 2.0
    \CorrectChoice 2.5
    \choice It does not print, it will throw an error
\end{choices}

\question[2] Which of the following is "a name for a location in memory used to hold a data value"?
\begin{choices}
    \choice literal
    \CorrectChoice variable
    \choice declaration statement
    \choice assignment statement
\end{choices}

\question[2] Which of the following is the correct binary form of 41?
\begin{choices}
    \CorrectChoice 101001$_2$
    \choice 100101$_2$
    \choice 101000$_2$
    \choice 100100$_2$
\end{choices}

\question[2] Which of the following is the correct decimal form for the result of 111$_2$ AND 101$_2$?
\begin{choices}
    \choice 1
    \choice 2
    \choice 4
    \CorrectChoice 5
\end{choices}

\question[2] Which of the following lines of code will print "5.0" without an error? 
\begin{choices}
    \CorrectChoice \ic{System.out.printf("\%d", 5.0)}
    \choice \ic{System.out.printf("\%f", 5.0)}
    \choice \ic{System.out.print("\%f", 5.0)}
    \choice \ic{System.out.print("\%d", 5.0)}
\end{choices}

\question[2] What is the purpose of the Java Virtual Machine? 
\begin{choices}
    \choice Help display graphics generated by Java code. 
    \CorrectChoice Make it easier to share compiled code. 
    \choice Help organize and write Java code. 
    \choice Speed up runtime of Java code. 
\end{choices}

\question[2] What command can be used to display the contents of the current working directory in the terminal?
\begin{choices}
    \choice cat
    \CorrectChoice ls
    \choice touch
    \choice cp
\end{choices}

\question[2] Which of the following is not one of the four layers of the TCP/IP?
\begin{choices}
    \choice application layer
    \CorrectChoice packaging layer
    \choice transport layer
    \choice internet layer
\end{choices}

\question[2] What are the two kinds of messages used in a handshake? 
\begin{choices}
    \choice "data transfer" and "data reception"
    \CorrectChoice "synchronization" and "acknowledgement"
    \choice "upload" and "download"
    \choice "on" and "off" 
\end{choices}

\question[2] In a network with a star topology, what is the maximum number of hops a message would travel between any two computers? (Hint: draw a star topology to help visualize the problem) 
\begin{choices}
    \choice 0
    \choice 1
    \CorrectChoice  2
    \choice 3
\end{choices}

\question[2] What does LAN stand for? 
\begin{choices}
    \choice Large Area Network
    \CorrectChoice Local Area Network
    \choice Large Arrayed Network
    \choice Local Arrayed Network
\end{choices}

\section{Short Answer}

\question[5] Is there any time an "else if" statement can accomplish that nested "if" and "else" statements could not? If yes, provide an example. If not, why does Java include an "else if" statement? 

\begin{solution}
The "else if" statement is not necessary, we could always use nested "if" and "else" statements. However, the "else if" statement can make code significantly shorter and easier to read. 
\end{solution}

\question[5] What is the difference between Main Memory (RAM) and Secondary Memory? 

\begin{solution}
Main memory is volatile, meaning that the contents of the memory is not preserved when a computer is turned off and back on. On the other hand Secondary Memory is meant to be persistent or nonvolatile, and does not go away when the computer is turned off.
\end{solution}

\question[5] On the terminal, you are in a directory called \ic{/users/jdoe/scans} which contains many files. You want to list all files in the directory, then move all of \ic{.pdf} files to \ic{/users/jdoe}, and finally delete all the files remaining in \ic{/users/jdoe/scans}. What three commands would you use?

\begin{solution}
    Use \ic{ls}, followed by \ic{mv *.pdf ..}, and finally \ic{rm *}.
\end{solution}

\question Indicate what each of the following code snippets will print out.
\begin{parts}
    \part[2] \hfill\\
    
\begin{code}
int x = 7;
System.out.printf("%d\n7%dHi%d",x,x,x);
\end{code}

    \begin{solution}
    
        \ic{7}
        
        \ic{77Hi7}
    \end{solution}

    \part[2] \hfill\\
    
\begin{code}
boolean x = False;
boolean y = False;
if(x && y) {
    System.out.println("foo");
}
else if(x || y) {
    System.out.println("bar");
}
else {
    System.out.print("far");
}
\end{code}

    \begin{solution}
        \ic{far}
    \end{solution}

    \part[2] \hfill\\
    
\begin{code}
    int x = 50;
    int y = 27;
    if (x % 2) {
    	if (y % 3) {
	    System.out.print("foo");
	}
	else {
	    System.out.print("bar");
	}
    }
    else {
    	System.out.print("far");
    }
    
\end{code}

    \begin{solution}
        foo
    \end{solution}

    \part[4] \hfill\\

\begin{code}
System.out.print("hello\n");
System.out.println("world");
System.out.print("how");
System.out.printf("\nare %s?",you);
\end{code}

    \begin{solution}
        \hfill\\
        \ic{hello\newline world\newline how\newline are you?}
    \end{solution}
\end{parts}

\question Perform each of the following binary operations.
\begin{parts}
    \part[5] $11001_2 + 1100_2$
    \answerline[$100101_2$]
    \part[5] $1011_2 \,\&\; 10010_2$
    \answerline[$10_2$]
    \part[5] $10000_2 \,\|\; 11001_2$
    \answerline[$1001_2$]
\end{parts}

\section{Programming}

\question[15] Complete the Java program below to successfully ask a user for their name and then output "Greetings " followed by the name. For example, if a user types in "Alice" then the program should print "Greetings Alice". 

\begin{code}
import java.util.Scanner;

class GreetingsUser {
    public static void main(String[] args) {
    	Scanner input = new Scanner(System.in);
	System.out.print("Enter your name: ");
   	/* *************************** */

 
 
 
 
 
 
 	/* *************************** */
 	input.close(); 
    }
}
\end{code}

\begin{solution}
\begin{code}
String name = input.next(); // get next user input as a String
System.out.printf("Hello %s\n", name); 
\end{code}
\end{solution}

\question[15] Complete the Java program below to successfully ask a user for a number and then print either "even" if the number is even or "odd" if the number is odd. For example, if the user types in 37 then the program should print "odd"

\begin{code}
import java.util.Scanner;

class EvenOrOdd {
    public static void main(String[] args) {
    	Scanner input = new Scanner(System.in);
	System.out.print("Enter your number: ");
   	/* *************************** */

 
 
 
 
 
 
 	/* *************************** */
 	input.close(); 
    }
}
\end{code}

\begin{solution}
\begin{code}
int number = input.nextInt(); // get next user input as an int
if (number % 2 == 0) {
    System.out.print("even");
}
else {
    System.out.print("odd");
}
\end{code}
\end{solution}

\end{questions}

\end{document}