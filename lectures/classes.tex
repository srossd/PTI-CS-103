\chapter{Classes}
\label{ch:classes}

\begin{goals}
\item Understand classes and objects.
\item Understand class methods and attributes.
\item Know how to create objects from classes.
\item Write and compile programs with multiple classes.
\item Understand the \ic{public} and \ic{private} access-level modifiers.
\item Understand the \ic{static} modifier.
\end{goals}

Java is an \emph{object-oriented programming} (OOP) language, meaning that nearly everything is a ``class'' or an ``object''. While you may not have thought about it, you have been working with classes and objects this whole time. Now let's define these terms:

\begin{definition}
An \emph{object} is a combination of variables, functions, and data (state and behavior). A \emph{class} is a blueprint for a type of object.
\end{definition}

For example, if a car is an object, then a class is a car factory. Every car (object) that is made by the car factory (class) follows the basic car blueprint: it has tires, an engine, a steering wheel, and so forth. The type of an object is its class. So the type of an object from a car factory might be \ic{Car}.

As another example, here is a class that is very similar to classes you saw in Chapter~\ref{ch:methods}:

\begin{code}
/* HelloWorld.java */

class HelloWorld {

    static void greeting() {
        System.out.println("Hello World");
    } 

    public static void main(String[] args) {
        greeting();
    }
}
\end{code}

Every program in Java follows this basic pattern. You have a class, for example \ic{HelloWorld}. The name of the file must match the name of the class, for example \ic{HelloWorld.java}. Methods such as \ic{greeting} are associated with the class. And every Java program needs one \ic{main} method that gets executed first. In this chapter, we'll discuss classes in more detail. This will allow us to build more complex and interesting programs.

\section{Creating objects}

We've been working with classes this whole time. And in fact, you've also already seen objects. Consider this program from Chapter~\ref{ch:io}:

\begin{code}
import java.util.Scanner;

class HelloPlanet {

    public static void main(String[] args) {
        // Create a Scanner object.
        Scanner input = new Scanner(System.in);
        System.out.println("Enter a planet name: ");
        String planet = input.next();
        System.out.printf("Hello %s\n", planet);
        input.close();  // Close the Scanner object.
    }
}
\end{code}

Notice that on Line 7, we create an object \ic{input} using the \ic{new} keyword. We say that \ic{input} has type \ic{Scanner}. Again, nearly everything in Java is an object, so a similar thing is happening on line 9: \ic{String planet = ...} is creating a \ic{planet} object of type \ic{String}. This object is the user's input, e.g. "Jupiter".

Since a class is like a factory for objects of a given type, this means we can create multiple objects from a single class. Consider this example of two objects of type \ic{Dog}:

\begin{code}
class Dog {

    void bark() {
        System.out.println("Woof, woof!");
    }
    
    public static void main(String[] args) {
        Dog fido = new Dog();
        fido.bark();  // Prints "Woof, woof!"
        Dog izzy = new Dog();
        izzy.bark();  // Prints "Woof, woof!"
        // False; they are not the same object.
        System.out.println(fido == izzy);
    }
}
\end{code}

The important point is that \ic{fido} and \ic{izzy} are both \ic{Dog} objects. They both have a \ic{bark} method. But they're distinct objects.

\section{Class members}

We have seen that classes have methods. For example, this snippet calls the \ic{Scanner} class's \ic{nextInt} method:

\begin{code}
Scanner input = new Scanner(System.in);
int num = input.nextInt();
\end{code}

However, classes also have \emph{attributes}. An attribute, sometimes called a \emph{field}, is a variable associated with a class. We refer to both methods and attributes as \emph{members}. Every object constructed from the class will then have the same variable name, although not necessarily the same value. For example, we can modify the \ic{Dog} class to have a \ic{color} attribute:

\begin{code}
class Dog {

    // Un-initialized `color` attribute.
    String color;

    void bark() {
        System.out.println("Woof, woof!");
    }
    
    public static void main(String[] args) {
        Dog fido = new Dog();
        fido.color = "brown";
        Dog izzy = new Dog();
        izzy.color = "black";
        System.out.println(fido.color);  // Prints "brown"
        System.out.println(izzy.color);  // Prints "black"
    }
}
\end{code}

Notice that we access the attributes like we access methods, using dot notation. So while both \ic{Dog} objects have an \ic{age} attribute, that attribute is different for each instance. Setting Fido's color does not effect Izzy's color, although both dogs have colors.

Furthermore, a method can depend on the value of an attribute. Therefore, the runtime behavior of an object may depend on the object's state. For example

\begin{code}
class Dog {

    int age = 0;
    
    void bark() {
        System.out.printf("Woof! I'm %d-years-old.\n", age);
    }
    
    public static void main(String[] args) {
        Dog fido = new Dog();
        fido.age = 3;
        Dog izzy = new Dog();
        izzy.age = 5;
        fido.bark();  // Prints "Woof! I'm 3-years-old."
        izzy.bark();  // Prints "Woof! I'm 5-years-old."
    }
}
\end{code}

Notice that the result of calling the method \ic{bark} can change depending on the object's state.  This is what we mean when we say, ``Java is an object-oriented programming language.'' Nearly everything in Java is actually a class or an object, and we build complex programs by composing objects and by manipulating them and their states.

\section{Constructors}

We may want to set attributes such as Izzy's age when we create the object. We can do this using a constructor. In Java, a \emph{constructor} is a special method that initializes the object. The constructor is called when the class is created. For example:

\begin{code}
class Dog {

    // Declare but do not initialize class attributes.
    String color;
    int age;
    
    // This is a class constructor for the `Dog` class.
    public Dog(String myColor, int myAge) {
        color = myColor;
        age = myAge;
    }
    
    void bark() {
        System.out.printf("Woof! I have %s fur and %d-years-old.\n", age);
    }
    
    public static void main(String[] args) {
        Dog fido = new Dog("brown", 3);
        Dog izzy = new Dog("black", 5);
        // Prints "Woof! I have brown fur and 3-years-old."
        fido.bark();
        // Prints "Woof! I have black fur and 5-years-old."
        izzy.bark();
    }
}
\end{code}

As we can see, the \ic{Dog} constructor is a bit like a method in that it can accept parameters, in this case \ic{myColor} and \ic{myAge}. However, it is only called once per object using the \ic{new} keyword.

\section{Creating multiple classes}

Since every Java class must go into a separate file, we can create programs with multiple classes by creating separate files with their respective class definitions and then compiling those files together.

Let's look at an example. Consider these two classes, \ic{Car} and \ic{CarDealership}:

\begin{code}
// Car.java

class Car {
  
  String make;
  double price;
 
  public Car(String myMake, double myPrice) {
    make = myMake;
    price = myPrice;
  }
}
\end{code}

\begin{code}
// CarDealership.java

class CarDealership {
  
  public static void printPrice(Car car) {
    System.out.printf("%s costs: $%1.2f\n", car.make, car.price);
  }
    
  public static void main(String[] args) {
    Car car1 = new Car("Toyota", 12000);
    Car car2 = new Car("Ford", 8000);
    printPrice(car1);
    printPrice(car2);
  }
}
\end{code}

The \ic{Car} class represent car objects. Its constructor takes a string, for the car's make or brand name, such as Lincoln, Ford, and Chevrolet, as well a price. The \ic{CarDealership} class uses the \ic{Car} class to print information about each car. 
We can compile these two files together by passing both filenames to the \ic{javac} command:

\begin{monospace}
$ javac Car.java CarDealership.java
\end{monospace}

Notice that between these two files, there is only one \ic{main} method. This represents the start of the program. If we try to run the compiled \ic{Car} program, we will get an error:

\begin{monospace}
$ java Car
Error: Main method not found in class Car, please define the main method as:
   public static void main(String[] args)
or a JavaFX application class must extend javafx.application.Application
\end{monospace}

This error tells us that the \ic{java} executable cannot find the \ic{main} method to start the program. Instead, we need to run the \ic{CarDealership} program:

\begin{monospace}
$ java CarDealership
Toyota costs: $12000.00
Ford costs: $8000.00
\end{monospace}

We can imagine more complex programs. For example, here is a program that computes the average price of the cars in the dealership:

\begin{code}
class CarDealership {
  
  public static void printAveragePrice(Car[] cars) {
    double avg = 0;
    for (int i = 0; i < cars.length; i++) {
      avg += cars[i].price;
    }
    avg = avg / cars.length;
    System.out.printf("Average price: %1.2f\n", avg);
  }
    
  public static void main(String[] args) {
    Car[] cars = new Car[3];
    cars[0] = new Car("Toyota", 12000);
    cars[1] = new Car("Ford", 8000);
    cars[2] = new Car("Lincoln", 13500);
    printAveragePrice(cars);
  }
}
\end{code}

After compiling this program and running it, we get

\begin{monospace}
$ java CarDealership
Average price: 11166.67
\end{monospace}

Now that we understand methods, arrays, and multiple class programs, we can really start to build cool stuff.

\section{Packages}

Notice in the previous \ic{HelloPlanet} program, we import something called \ic{java.util.Scanner}. In fact, we've been using the \ic{import} keyword since nearly the beginning without explaining it, e.g.:

\begin{code}
import java.util.Scanner;
\end{code}

This \ic{import} statement tells the Java compiler that we want to include the \ic{Scanner} class from the \ic{java} package. What's a \emph{package}?

\begin{definition}
A \emph{package} organizes a set of related classes under a common name or \emph{namespace}.
\end{definition}

We can think of a package as just a folder of class files. In programming, this kind of organizational technique is called \emph{namespacing}. We are grouping or categorizing parts of our program under a specific name, which is analogous to a folder or path.

For example, you can think of \ic{java.util.Scanner} as referring to a file \ic{Scanner.java} in the \ic{util} subdirectory of the \ic{java} root directory. As a programmer, you can write your own packages for others to use. In this course, you won't learn to do that, but you should know that packages exist and that we use code from others by importing packages.

See \ref{appendix:packages} for a list of useful packages.

\section{Encapsulation}

Now that we can build programs with multiple classes, we need to talk about a concept called ``encapsulation'':

\begin{definition}
\emph{Encapsulation} is the concept of restricting access to private data.
\end{definition}

Why might this be useful? As the programs you write become bigger and bigger, you'll find it harder and harder to reason about what is happening or the state of a given object. Encapsulation helps manage complexity because it limits how different parts of a program can interact.

\marginnote{MIT professor Hal Abelson said: "The real problems [of computer science] come when we try to build very, very large systems, computer programs that are thousands of pages long, so long that nobody can really hold them in their heads all at once. And the only reason that that's possible is because there are techniques for controlling the complexity of these large systems."}

This point is fairly abstract. Let's look at a concrete example:

\begin{code}
class Person {
  // The "private" keyword means the attribute cannot be
  // accessed outside of the class.
  private String name;

  public String getName() {
    return name;
  }

  public void setName(String newName) {
    this.name = newName;
  }
}
\end{code}

We can get the value of the person's name using the \ic{getName} function, called a \emph{getter} function; and we can set the value of the person's name using the \ic{setName} function, called a \emph{setter} function. However, we cannot access this property directly:

\begin{code}
class MyClass {
  public static void main(String[] args) {
    Person p = new Person();
    p.name = "Dave";  // Error: name has private access in Person.
  }
}
\end{code}

What's happening? We declared the \ic{name} attribute on the \ic{Person} class to be "private" using the \ic{private} keyword. This means that the attribute cannot be directly accessed outside of the class. Instead, we must use the getter/setter pattern to access the private data.

\begin{definition}
We call the keywords \ic{public}, \ic{private}, and \ic{protected} \emph{access-level modifiers}. Access-level modifiers control how members (attributes and methods) and classes can be accessed:
\begin{itemize}
    \item A \ic{public} member or class is accessible to all classes and packages.
    \item A \ic{protected} member or class is only accessible within its own package.
    \item A \ic{private} member is only accessible within the same class as it is declared. We won't discuss \ic{private} classes.
    \item A member with \emph{no access modifier} is only accessible to classes within the same package.
\end{itemize}
\end{definition}

Why are access-level modifiers useful? What's the point of making \ic{name} private and yet allowing the user to modify name via getters and setters? Encapsulation using access-level modifiers means the programmer has better control of class attributes and methods. This concept is easier to understand if we imagine multiple programmers working together.

\subsection{A tale of two programmers}

\marginnote{In programming, the \emph{front end} refers to the presentation layer or user interface, and the \emph{back end} refers to the data layer. For example, in an online store, the front end is the UI or website, while the back end is the server, which handles inventory and processes orders.}

Imagine that two programmers, Alice and Bob, are building the back end of an online shopping website, called ``Nile''. Alice's code is responsible for processing orders, while Bob's code is responsible for managing inventory. Alice implements a package called \ic{nile.transactions}, while Bob implements a package called \ic{nile.inventory}. Both of them need to deal with objects that represents items, so they agree upon the following class:

\begin{code}
class Item {
  private double price;
  
  public double getPrice() {
    return price;
  }
  
  protected void setPrice(double newPrice) {
    if (newPrice < 0) {
      throw new java.lang.Error("A item's price cannot be negative.");
    } else {
      price = newPrice;
    }
  }
}
\end{code}

Since Bob manages the inventory, he writes the \ic{Item} class in his \ic{nile.inventory} package. Since \ic{setPrice} is protected, only his code can modify the price by calling the \ic{setPrice} function. Any code within Bob's package can set the price of an item, but Alice's code, which simply import's Bob's package, cannot set the price.

Why does Bob use a setter function? Since \ic{price} is just a double, it can be set to any value, positive or negative. However, Bob knows that \ic{price} represents the cost of an item in the Nile warehouse. He wants to ensure that he never accidentally sets the price to a zero or a negative number. By implementing a setter function, \ic{setPrice}, Bob ensures he can always error handle new prices whenever he needs to set them. Without this functionality, Bob would have to do error handling every wherever he explicitly sets the price.

\marginnote{In computer science, \emph{factoring} means breaking a problem or system into easier-to-manage parts. When Alice and Bob agree to divide their work into processing orders and managing inventory, they are factoring or decomposing their bigger problem---building an online ordering system.

\emph{Refactoring} is the process of restructuring existing code.}

Furthermore, encapsulation limits the amount of code you need to refactor. For example, imagine that Alice and Bob agree that Nile will be an English-only website. However, they hope that one day, they may expand Nile to serve Spanish-speakers. They decide that each item will have a \ic{description} member which is accessible via a getter function, and that the getter function will take a \ic{lang} argument:

\begin{code}
class Item {
  
  private double price;
  private String description;
  
  public double getPrice() {
    return price;
  }
  
  private String getDescription(String lang) {
    if (lang == "English") {
      return description;
    } else {
      String msg = "'English' is the only supported value for 'lang'.";
      throw new UnsupportedOperationException(msg);
    }
  }
  
  protected void setPrice(double newPrice) {
    if (newPrice < 0) {
      throw new java.lang.Error("A item's price cannot be negative.");
    } else {
      price = newPrice;
    }
  }
}
\end{code}

Why is this nice? Alice can immediately start using Bob's \ic{Item} class, calling \ic{getDescription} with "English" as the argument. If she uses the wrong \ic{lang} argument, the method tells her what she needs to change. However, if Bob's inventory is ever refactored to support Spanish or another language, Alice just needs to call \ic{getDescription} with the desired language. In this way, encapsulation limits the scope of what needs to be changed when improving or modifying a code base.

\subsection{The \ic{static} modifier}

Sometimes, we want to create class members that are common to the class, not the objects. We can do this using the \ic{static} modifier. Consider this code:

\marginnote{Now that we understand \ic{public} and \ic{static}, we can understand the function declaration \ic{public static void main(String[] args)}. This function is public, meaning accessible outside of its class file; it is static, meaning you do not have to instantiate the class in order to call the method; its return type is \ic{void}, meaning it has no return value; and it takes an array of \ic{String} objects as an argument. Calling this method with the appropriate arguments is handled by the program that runs your Java program!}

\begin{code}
class Bicycle {
  
    private String type;
    private int speed;
    private static int numBikes = 0;
    
    public Bicycle(String newType, int newSpeed) {
      type = newType;
      speed = newSpeed;
      numBikes += 1;
    }
    
    public static void main(String[] args) {
      Bicycle x = new Bicycle("Road", 10);
      Bicycle y = new Bicycle("Commuter", 1);
      Bicycle z = new Bicycle("Mountain", 12);

      System.out.println(x.numBikes);  // Prints 3
      System.out.println(y.numBikes);  // Prints 3
      System.out.println(z.numBikes);  // Prints 3
    }
}
\end{code}

Notice that each \ic{Bicycle} object (\ic{x}, \ic{y}, and \ic{z}) have the same attribute value for \ic{numBikes}. This is because \ic{numBikes} is a static member and is therefore associated with the class, not the objects themselves. Without the \ic{static} keyword, we would expect each \ic{println} statement to print 1 rather than 3.

\exercisesection

\begin{exercise}
Create a class named \ic{MyClass} by completing the snippet below.  Save this code to a file (what should the filename be?) and ensure it compiles.

\begin{code}
public class ...
\end{code}
\end{exercise}

\begin{exercise}
Add the \ic{main} method to \ic{MyClass}. What is the argument to \ic{main}? (You've seen this many times by now!)
\end{exercise}

\begin{exercise}
In the \ic{main} method, create a new object of type \ic{MyClass} named \ic{myObj}. It should look like:

\begin{code}
____ ____ = new ____();
\end{code}
\end{exercise}

\begin{exercise}
Create a class attribute named \ic{x} and then print it in \ic{main}:

\begin{code}
public class MyClass {
  
  // Create an integer with a value 5 here.

  public static void main(String[] args) {
    MyClass myObj = new MyClass();
    // Add a print statement here.
  }
}
\end{code}
\end{exercise}

\begin{exercise}
Create a method named \ic{myMethod} on the object. It should just print "My method!" What should its return type be? Call \ic{myMethod} in \ic{main}.

\begin{code}
public class MyClass {
  
  // Create a public method `myMethod` here.
  // It should print "My method!"

  public static void main(String[] args) {
    MyClass myObj = new MyClass();
    // Call `myMethod` here.
  }
}
\end{code}
\end{exercise}

\begin{exercise}
Fill in the missing parts to import the \ic{Scanner} package:

\begin{monospace}
______ ______.______.______;
\end{monospace}
\end{exercise}