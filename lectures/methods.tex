\chapter{Methods}\label{ch:methods}

\begin{goals}
\item Write methods with parameters and non-void return types
\item Understand the components of a method
\item Study the main method in the context of the lesson
\end{goals}

% Abstraction, modularity, code-reuse, syntax, arguments and calling

% Sometimes it is useful to abstract certain computations into methods.
% In this chapter, we will be looking at how and why we can use methods to
% achieve more concise and easily-maintainable code.

Let's do a quick review of how to write a program in Java. Every line of code that runs in Java must be inside a class. A class should always start with an uppercase letter, and the name of the java file should be the same as the classname but with ``.java" at the end. 

Every Java program needs a \ic{main()} method that looks like this:

\begin{code}
public static void main(String[] args)
\end{code}

For example, our hello world program from the beginning of the semester is 

\begin{code}
class MyClass {
  public static void main(String[] args) {
    System.out.println("Hello World");
  }
}
\end{code}

Any code inside the \ic{main()} method is executed when the program is run. But after all this time, we still don't know what a method is! Let's fix that. 

\section{Methods}

\subsection{Introduction}

A \emph{method} is a block of code which only runs when it is called. You can pass data, known as parameters, into a method. Methods are used to perform certain actions, and they are also known as \emph{functions}.

We use methods in order to define code once, and use it many times. They help organize code and make debugging much faster. We will understand the benefits more clearly in coming lectures and exercises. 

A method must be declared within a class. It is defined with the name of the method, followed by parentheses (). Java provides some pre-defined methods, such as \ic{System.out.println()}, but you can also create your own methods. Here is an outline for making a method named ``myMethod" within a class called ``MyClass": 

\begin{code}
class MyClass {
  public static void myMethod() {
    // code to be executed
  }
}
\end{code}

\ic{myMethod()} is the name of the method. Public means that code outside of the MyClass class will be able to run the method. Static means that the method belongs to the MyClass class and not an object of the MyClass class. We won't explain in detail exactly these definitions of "public" and "static" mean until later in the course. For now, you should just remember to write ``public static" at the start of every method declaration. \ic{void} means that this method does not have a return value. We will understand what this means later in this chapter. 

Now let's define what the method should do. Let's make our method print \ic{Hello World}. 
\begin{code}
class MyClass {
  public static void myMethod() {
    System.out.println("Hello World");
  }
}
\end{code}

We've successfully written a method! But remember that every Java program needs a \ic{main()} method, since the \ic{main()} method is what Java tries to execute when you run the program. Let's see the error Java throws if we \emph{don't} have a main method. Save this file as ``MyClass.java" (filename must match class name!) within a new folder called ``Methods" in your ``CS103" folder. Now compile, and try running the program. You should see the error \ic{Static Error: This class does not have a static void main method accepting String[].} Big lesson: only code within the main method actually gets executed. If we want to run the code in the \ic{myMethod()} function, we need to \emph{call} the \ic{myMethod()} method from the \ic{main()} method. To call a method in Java, write the method's name followed by two parentheses (). As always, your statement should end in a semicolon. The full program should look like this: 

\begin{code}
class MyClass {
  public static void myMethod() {
    System.out.println("Hello World");
  }

  public static void main(String[] args) {
    myMethod();
  }
}
\end{code}

Compile and run this code, and you should see the output ``Hello World". 

Methods can be run multiple times. Try calling the \ic{myMethod} method 3 times from the \ic{main()} method:\marginnote{It's often helpful to trace through the code with your finger to understand the order of execution. Although \ic{myMethod()} comes before the \ic{main()} method in this program, when you run a program Java always just executes the \ic{main()} method. Try stepping through this example with your finger, showing the order in which Java looks at each piece of the program. What would happen if we had no code in our \ic{main()} method?}

\begin{code}
class MyClass {
  public static void myMethod() {
    System.out.println("Hello World");
  }

  public static void main(String[] args) {
    myMethod();
    myMethod();
    myMethod();
  }
}
\end{code}

This should print ``Hello World" 3 times, separated by newlines. 

You can define as many methods as you want in a Java program as long as they have different names. Let's see this with an example: 

\begin{example}
Fill in the blanks within the main method of the following code to make it print ``Hello" then ``World" by calling the two methods \ic{printHello} and \ic{printWorld}. 
\begin{code}
class MyClass {

    public static void printHello() {
        System.out.print("Hello ");
    }
    
    public static void printWorld() {
        System.out.println("World");
    }

    public static void main(String[] args) {
        ______________ // call method to say "Hello"
        ______________ // call a method to say "World"
    }
}
\end{code}
\noindent \emph{Answer}:
\begin{code}
    printHello();
    printWorld();
\end{code}
\end{example}

\marginnote{Not only can you have as many methods as you want in a Java program, you can also put them in any order. Try typing the above into DrJava and make sure it runs and outputs what you expect. Then try moving the \ic{main()} method before the \ic{printHello()} and \ic{printWorld()} methods. Ensure it runs exactly the same.}

\subsection{Parameters} 

Information can be passed to methods as \emph{parameters}. Parameters act as variables inside the method. Parameters are specified after the method name, inside the parentheses. You can add as many parameters as you want, just separate them with a comma.

The following example has a method called \ic{printFullName} that takes a String called \ic{firstName} as parameter. 

\begin{code}
class MyClass {
  public static void printFullName(String firstName) {
    System.out.println(firstName + " Reyes");
  }

  public static void main(String[] args) {
    printFullName("Liam");
    printFullName("Jenny");
    printFullName("Anja");
  }
}
\end{code}

Save, compile, and try running it. 

\marginnote{Note that we can name parameters whatever we want. How would you rewrite the \ic{printFullName()} method to use the parameter name \ic{foreName} instead of \ic{firstName}?}When a parameter is passed to the method, it is called an \emph{argument}. So, from the example above: fname is a \emph{parameter}, while Liam, Jenny and Anja are \emph{arguments}. When the method is called, we pass along a first name as an argument and the method prints out the person's name with a last name of ``Reyes" appended. When we pass in ``Liam" as an argument to the \ic{printFullName()} method, Java makes a new variable called \ic{firstName} to be used within the \ic{printFullName()} method body. So whenever we see \ic{firstName} during the method call with argument ``Liam", Java really thinks of it as the String ``Liam". When we pass in the argument ``Jenny", Java sets \ic{firstName} to ``Jenny". 

Now let's try an example with multiple parameters:

\begin{example}

Fill in the two blanks in the code below to print the message ``Liam is 5" ``Jenny is 8" and ``Anja is 31"
\begin{code}
class MyClass {
  public static void printNameAndAge(String firstName, int age) {
    System.out.println(firstName + " is " + ____);
  }

  public static void main(String[] args) {
    printNameAndAge("Liam", 5);
    printNameAndAge("Jenny", 8);
    // call a method to print the message "Anja is 31":
    ____________________ 
  }
}
\end{code}
\end{example}

\marginnote{What are the parameters and what are the arguments in the above example calling the \ic{printNameAndAge()} example?} Note that when you are working with multiple parameters, the method call must have the same number of arguments as there are parameters, and the arguments must be passed in the same order.

\subsection{Return types}
The void keyword, used in the examples above, indicates that the method should not return a value. If you want the method to return a value, you can specify a data type (such as int, char, etc.) instead of void, and use the return keyword\marginnote{when we write ``return" followed by a space then an expression, the method will immediately spit out the expression so it can be used by whatever code called the method} inside the method: 

\begin{code}
class MyClass {
  public static int addFive(int x) {
    return 5 + x;
  }

  public static void main(String[] args) {
    System.out.println(addFive(3));
  }
}
\end{code}

Here's another example with a method taking two arguments:

\begin{code}
class MyClass {
  static int takeSum(int x, int y) {
    return x + y;
  }

  public static void main(String[] args) {
    System.out.println(takeSum(5, 3));
  }
}
\end{code}

\marginnote{What does the takeSum method do? What is the output of the full example program?}

Whereas our previous examples printed information from within our methods, now we are getting the output of a method in the main method and printing its value from there. 

Non-void return types also allow us to store results from methods in variables. Using variables to store the return value from methods is recommended, since it makes code easier to read and maintain. In the following example, we store the return value of \ic{takeSum(5, 3)} in the integer \ic{z}:

\begin{code}
class MyClass {
  static int takeSum(int x, int y) {
    return x + y;
  }

  public static void main(String[] args) {
    int z = takeSum(5, 3);
    System.out.println(z);
  }
}
\end{code}

\subsection{Endless possibilities} 

All of the concepts we've learned so far can be applied in methods, from control structures to loops to arrays. Here's an example of a method that makes use of if...else: \marginnote{Note that in all of the examples so far our methods start with a verb: the method name says what the method does. ``printHello" printed ``Hello", ``printFullName" printed a person's full name, ``addFive" added five to a number, ``checkAge" checked a person's age to see whether to grant access, and ``getSmaller" got the smaller of two numbers. Try to always start with a verb when you're deciding how to name your methods. For example, what might you call a method meant to take a number from 0-100 as argument and return the number's corresponding letter grade from A-F?}

\begin{code}
class MyClass {

  // Create a checkAge() method with an integer variable called age
  public static void checkAge(int age) {

    // If age is less than 18, print "access denied"
    if (age < 18) {
      System.out.println("Access denied - You are not old enough!");

    // If age is greater than 18, print "access granted"
    } else {
      System.out.println("Access granted - You are old enough!");
    }

  }

  public static void main(String[] args) {
    checkAge(20); // Call the checkAge method and pass along an age of 20
  }
}
\end{code}

What does the code above output?

\begin{example}
Fill in the blanks in the code below to make a method that returns the smaller of two numbers:

\begin{code}
class MyClass {

  public static int getSmaller(int num1, int num2) {
    int ret;
    if (num1 < num2) {
        ret = ____;
    }
    else {
        ret = ____;
    }
    return ret; 
  }

  public static void main(String[] args) {
    System.out.println(getSmaller(1,2));
  }
}
\end{code}

\noindent \emph{Answer}:

\textbf{1st blank:} num1

\textbf{2nd blank:} num2
\end{example}

Let's use loops!

\begin{example}
Fill in the blanks in the code below to make a method that adds all numbers in an array and calls it on \ic{\{1, 2\}} to return \ic{3}. 

\begin{code}
class MyClass {
  // make sure the parameter numArray's type is integer array
  public static int addNumbers(____ numArray) {
    int sum = 0;
    for (int i=0; i<numArray.length; i++) {
      sum = sum + numArray[i]; // print i
    }
    return sum; 
  }

  public static void main(String[] args) {
    int[] inputArray = {1,2};
    int sum = ______; // call the addNumbers method on inputArray
    System.out.println(sum);
  }
}
\end{code}

\noindent \emph{Answer}:

\textbf{1st blank:} int[] 

\textbf{2nd blank:} addNumbers(inputArray)
\end{example}


% We will first look at some code that does not use methods and identify some
% problems with this code. We will then, in Section \ref{sec:methods},
% introduce what a method is, including how to define and use them.
% In Sections \ref{sec:abstraction} and \ref{sec:modularity}, we will
% then talk about how we can use methods to improve the problems that
% we noticed with our code. In particular, in Section \ref{sec:abstraction},
% we talk about how methods allow us to abstract some of our code to get
% more concise and maintainable code, and in Section \ref{sec:modularity},
% we talk about how methods allow us to divide our code up into chunks
% that we can consider in isolation, allowing us to manage and maintain
% our code more easily. We'll finish the chapter by getting a more concrete understanding of how to write programs using methods. 

% Let us consider the following problem:

% We have a class of nine students, split up into three groups of three.
% We want to report the maximum score for each group.
% %as well as the maximum score for the students overall.
% We will use the variable names \ic{group1s1}, \ic{group1s2}, and \ic{group1s3}
% to refer to the three scores achieved by the students in Group 1, and we will use
% a similar naming convention for the variable names for the scores achieved by
% students in Group 2 and Group 3.

% The following snippet of code computes the maximum score in Group 1, storing
% the result in \ic{group1Max}.
% \begin{code}
% int group1Max = group1s1;
% if (group1s2 > group1Max) {
%   group1Max = group1s2; 
% }
% if (group1s3 > group1Max) {
%   group1Max = group1s3; 
% }
% \end{code}

% If we now wish to extend this to compute the maximum score for each group,
% we can copy the code and rename the variables to achieve the following
% code:
% \begin{code}
% int group1Max = group1s1;
% if (group1s2 > group1Max) {
%   group1Max = group1s2; 
% }
% if (group1s3 > group1Max) {
%   group1Max = group1s3; 
% }

% int group2Max = group2s1;
% if (group2s2 > group1Max) {
%   group2Max = group2s2; 
% }
% if (group2s3 > group2Max) {
%   group2Max = group2s3; 
% }

% int group3Max = group3s1;
% if (group3s2 > group3Max) {
%   group3Max = group3s2; 
% }
% if (group3s3 > group3Max) {
%   group3Max = group3s3; 
% }
% \end{code}

% \noindent There are a couple of things about the process of getting this resulting code
% that aren't particularly ideal.
% The first is that it is very easy to make a mistake when copying the code and renaming
% the variables. For example, we might have missed renaming \ic{group1Max} to \ic{group3Max}
% in the computation of the maximum score for Group 3, leading us to compute an incorrect maximum
% value in some cases.

% The second issue is that there is a lot of redundancy.
% If we look at the code, we can notice that there seem to be a lot of redundant parts
% across the code for the three different groups.
% If we had ended up with the wrong computation initially and needed to fix it
% later, then we would have to make sure we fixed it everywhere, leading to repeated work and the
% possibility that we didn't fix one of the copies. For example, if we initially started out with
% \ic{group1s3 < group1Max} instead of \ic{group1s3 > group1Max}, and then copied and renamed variables
% to compute the maximum scores for the other groups, we would end up with the code below:

% \begin{code}
% int group1Max = group1s1;
% if (group1s2 > group1Max) {
%   group1Max = group1s2;
% }
% if (group1s3 < group1Max) {
%   group1Max = group1s3;
% }

% int group2Max = group2s1;
% if (group2s2 > group1Max) {
%   group2Max = group2s2;
% }
% if (group2s3 < group2Max) {
%   group2Max = group2s3;
% }

% int group3Max = group3s1;
% if (group3s2 > group3Max) {
%   group3Max = group3s2;
% }
% if (group3s3 < group3Max) {
%   group3Max = group3s3;
% }
% \end{code}

% To fix our code, we would not only have to replace \ic{group1s3 < group1Max} with
% \ic{group1s3 > group1Max}, but we would also need to replace \ic{group2s3 < group2Max} with
% \ic{group2s3 > group2Max} and \ic{group3s3 < group3Max} with \ic{group3s3 < group3Max}.

% \section{Methods}\label{sec:methods}
% In Java, we can use abstractions in our code in the form of \emph{methods}.
% Methods contain bits of code that may compute things using \emph{arguments}
% that are passed into the method.

% An example of the syntax for a \emph{method definition} is given below:
% \begin{code}
% public bool negate(bool arg) {
%   return !arg; // method body
% }
% \end{code}
% We can identify the different components of the method definition:
% \begin{itemize}
% \item  The \emph{access modifier} of the method is \ic{public},
% \item the \ic{bool} before \ic{negate} gives the \emph{return type} of the method
% \item \ic{negate} is the \emph{name of the method},
% \item \ic{arg} is the \emph{name of the first argument/parameter} of the method,
% \item the \ic{bool} before \ic{arg} it is the \emph{type} of the \ic{arg} parameter,
% \item and the code between the curly braces is the method \emph{body}.
% \end{itemize}
% Inside the method body, there is code that may use the arguments of the method.
% In the example above, the body only consists of the \ic{return} statement.
% The \ic{return} statement is followed by something of the return type of the method.
% This expression, after being evaluated, gives the \emph{return value} of the method.
% As soon as a \ic{return} statement is encountered, a method finishes its execution.
% \emph{Access modifiers} will be described later on when we talk about Classes,
% but note that if an access modifier is omitted, it the method has the \emph{default} modifier.

% Other important terms to note are as follows:
% The \emph{parameter list} of a method consists of the types of its parameters in the
% order they are given in the method definition.
% For the above method, it is simply \ic{bool}.
% The \emph{method signature} consists of the method name and parameter list.
% For the above method, it is \ic{negate(bool)}.
% A method is identified by its method signature.

% \begin{example}
% Consider the following method:
% \begin{code}
% int squareSum(int arg0, int arg1) {
%   int res = arg0 + arg1;
%   return res * res;
% }
% \end{code}
% Give the following for the method:
% \begin{itemize}
% \item Access modifier
% \item Return type
% \item Arguments
% \item Method signature
% \end{itemize}

% \noindent \emph{Answer}: 
% \begin{itemize}
% \item Access modifier: default
% \item Return type: \ic{int}
% \item Arguments: \ic{arg0}, \ic{arg1}
% \item Method signature: \ic{squareSum(int, int)}
% \end{itemize}
% \end{example}

% \begin{example}
% Consider the following method:
% \begin{code}
% private void mystery (int age, String name) {
%   if (age >= 18) {
%     System.out.println(name);
%   }
% }
% \end{code} 
% Give the following for the method:
% \begin{itemize}
% \item Access modifier
% \item Return type
% \item Name
% \item Parameter list
% \end{itemize}

% \noindent \emph{Answer:}
% \begin{itemize}
% \item Access modifier: \ic{private}
% \item Return type: \ic{void}
% \item Name: \ic{mystery}
% \item Parameter list: \ic{int, String}
% \end{itemize}

% \end{example}

% We may \emph{call} or \emph{invoke} the above \ic{negate} method from elsewhere in the code by,
% for example, using \ic{negate(true)} if we want to call \ic{negate} with argument
% \ic{true}.
% We must specify the name of the method that we wish to call, followed by the arguments
% that we wish to call it on, separated by commas if there are more than one.
% The arguments provided in the call must correspond to the types of the arguments
% specified in the method definition: if the method definition has the type \ic{bool}
% for its first argument, then the first argument supplied in the call should also be
% of type \ic{bool}, and if the definition has the type \ic{int} for its second argument,
% then the first argument in the call should also be of type \ic{int}, and so on.
% A method invocation evaluates to the return value it computes, where this
% computation happens with the actual arguments
% given in the call substituted for the arguments specified in the definition, so in
% this case, the invocation \ic{id(true)} evaluates to \ic{false}.
% A method invocation has the same type as its return type, which in this case, is \ic{bool}.

% So, for example, we can do something like in following code snippet:
% \begin{code}
%   bool f = negate(true);
%   bool t = negate(negate(true));
% \end{code}
% After executing the above code, the variable \ic{f} contains the value \ic{false}
% and the variable \ic{t} contains the value \ic{true}.

% Note that we can also have methods that do not return anything. Such
% methods have \ic{void} return types. They may contain a \ic{return}
% statement with no expression following \ic{return}.
% Calls to these methods cannot be used inside of other expressions.
% Two examples of methods with \ic{void} return types are below:
% \begin{code}
% public void printSum(double a, double b) {
%   System.out.println(a + b);
% }
% \end{code}

% \begin{code}
% public void printBigger(double a, double b) {
%   if (a > b) {
%     System.out.println(a);
%     return;
%   }
%   System.out.println(b);
% }
% \end{code}
% Note that the latter example only ever prints the value of either \ic{a} or \ic{b} but
% never both because
% the \ic{return} statement finishes the execution of the method.

% Some methods also do not take any arguments at all. One such example is below:
% \begin{code}
% public int constantOne() {
%   return 1;
% }
% \end{code}
% \noindent This method can be called using \ic{constantOne()}.

% \begin{example}
% What value does \ic{res} have after the following code snippet is executed?
% \begin{code}
% int res = constantOne() + constantOne();
% \end{code}
% \emph{Answer}: The value of \ic{res} is 2.
% \end{example}.

% Note that for every method that does not have a \ic{void} return type,
% for every possible control-flow path through the method body,
% there must be a \ic{return} statement that is eventually reached.
% For example, the following method does not meet this requirement
% because when \ic{arg} is \ic{false}, the \ic{return} statement is not
% reached:
% \begin{code}
% public int twoIfTrue(bool arg) {
%   if (arg) {
%     return 2;
%   }
% }
% \end{code}

% \begin{example}
% Consider the following method:
% \begin{code}
% public void mystery (bool arg0, int arg1) {
%   if (arg0) {
%     return -1 * arg1;
%   }
%   return arg0;
% }
% \end{code}

% \noindent What do each of the following evaluate to?
% \begin{itemize}
% \item \ic{mystery(true, -2)}
% \item \ic{mystery(true, 0)}
% \item \ic{mystery(true, 4)}
% \item \ic{mystery(false, -2)}
% \item \ic{mystery(false, 0)}
% \item \ic{mystery(false, 4)}
% \end{itemize}

% \noindent \emph{Answer:}
% \begin{itemize}
% \item \ic{mystery(true, -2)} evaluates to 2
% \item \ic{mystery(true, 0)} evaluates to 0
% \item \ic{mystery(true, 4)} evaluates to $-4$
% \item \ic{mystery(false, -2)} evaluates to $-2$
% \item \ic{mystery(false, 0)} evalutes to 0
% \item \ic{mystery(false, 4)} evalutates to 4
% \end{itemize}

% \noindent What does the method do?

% \noindent \emph{Answer:} It negates its second argument
% \ic{arg1} whenever its first argument \ic{arg0} is
% \ic{true} and otherwise just returns its second argument
% \ic{arg1}.
% \end{example}

% \begin{example}
% How many methods can be called by the following method?
% What do you think each of their return types and parameter lists are?

% \begin{code}
% public void sayHello(int times, String name, String day) {
%   if (shouldStop(times))
%     return;
%   String hello = getHelloStr(name, day);
%   System.out.print.ln(hello);
%   sayHello(times - 1, name, day);
% }
% \end{code}
% \noindent \emph{Answer:}
% Three methods can be called: \ic{shouldStop}, \ic{getHelloStr} and \ic{sayHello}.

% The return type of \ic{shouldStop} is \ic{bool}, and it has parameter list \ic{int}.
% The return type of \ic{getHelloStr} is \ic{String}, and it has parameter list \ic{String, String}.
% The return type of \ic{sayHello} is \ic{void}, and it has parameter list \ic{int, String, String}.
% \end{example}

% \begin{example}
% Write a method with signature \ic{printInOrder(int, int, int)} that takes
% three \ic{int}s \ic{a}, \ic{b}, and \ic{c} as arguments and prints them out,
% one per line, in increasing order. E.g. \ic{printInOrder(5, 9, 7)} should output the following:
% \begin{code}
% 5
% 7
% 9
% \end{code}

% \noindent \emph{Answer:}
% A possible solution is the following:
% \begin{code}
% public void printInOrder(int a, int b, int c) {
%   int smallest = a;
%   int middle = b;
%   int biggest = c;
%   if (smallest > middle) {
%     int tmp = smallest;
%     smallest = middle;
%     middle = tmp;
%   }
%   if (smallest > biggest) {
%     int tmp = smallest;
%     smallest = biggest;
%     biggest = tmp;
%   }
%   if (middle > biggest) {
%     int tmp = middle;
%     middle = biggest;
%     biggest = tmp;
%   }
%   System.out.println(smallest);
%   System.out.println(middle);
%   System.out.println(biggest);
% }
% \end{code}
% \end{example}

\subsection{The main method}
To tie things together, let's go back to the main method for the Hello World program that we've seen so many times but never understood.

\begin{code}
class MyClass {
  public static void main(String[] args) {
    System.out.println("Hello World");
  }
}
\end{code}

In this case, the MyClass class has only one method, and it is the (required) main() method. ``public static" must be written for any method (at least for now). Main methods always have ``void" return types because Java simply runs them and doesn't store any values. And the parameter ``String[] args" is a parameter that always must be there for a main method. This String array will be filled with the words that you type after ``run MyClass" when running your program. We will get more practice with this in the next semester.  

\subsection{Overloading (Optional)}

With method \emph{overloading}, multiple methods can have the same name with different parameters. This is useful in a few cases.

For one, consider the following code that includes both a \ic{addTwoInts()} and a \ic{addTwoDoubles()} method so that we can add both integers and doubles in our \ic{main()} method. 

\begin{code}
public class MyClass{
  static int addTwoInts(int x, int y) {
    return x + y;
  }

  static double addTwoDoubles(double x, double y) {
    return x + y;
  }

  public static void main(String[] args) {
    int myNum1 = addTwoInts(8, 5);
    double myNum2 = addTwoDoubles(4.3, 6.26);
    System.out.println("int: " + myNum1);
    System.out.println("double: " + myNum2);
  }
}
\end{code}

Instead of defining two methods that should do the same thing (add two numbers), it is better to overload one. In the example below, we overload a single method called ``addNumbers" to work for both int and double:

\begin{code}
public class MyClass{
  static int addTwoNumbers(int x, int y) {
    return x + y;
  }

  static double addTwoNumbers(double x, double y) {
    return x + y;
  }

  public static void main(String[] args) {
    int num1 = addTwoNumbers(8, 5);
    double num2 = addTwoNumbers(4.3, 6.26);
    System.out.println("int: " + num1);
    System.out.println("double: " + num2);
  }
}
\end{code}



Because of this, we can actually have methods with the same name (in the same
Class) as long as they have different parameter lists.

For example, we can have the following two methods because
one has signature \ic{printBigger(double, double)} and the other
has signature \ic{printBigger(int, int)}:
\begin{code}
public void printBigger(double a, double b) {
  if (a > b) {
    System.out.println(a);
    return;
  }
  System.out.println(b);
}
public void printBigger(int a, int b) {
  if (a > b) {
    System.out.println(a);
    return;
  }
  System.out.println(b);
}
\end{code}
In this case, we say that the \ic{printBigger} method is \emph{overloaded}.

However, we cannot have the following two methods because they both have the
same signature \ic{twoIfTrue(bool)}:
\begin{code}
public int twoIfTrue(bool arg) {
  if (arg) {
    return 2;
  }
}
public double twoIfTrue(bool arg) {
  if (arg) {
    return 2.0;
  }
}
\end{code}

\begin{example}
Consider the following two methods that we have seen in previous
examples:
\begin{code}
private void mystery (int age, String name) {
  if (age >= 18) {
    System.out.println(name);
  }
}
public void mystery (bool arg0, int arg1) {
  if (arg0) {
    return -1 * arg1;
  }
  return arg0;
}
\end{code}
Are we allowed to have both of these methods together (in the same Class)?
Why or why not?

\noindent \emph{Answer:}
We are allowed to have both of these methods together because they have
different signatures. One has signature \ic{mystery(int, String)}
and the other has signature \ic{mystery(bool, int)}.
\end{example}

\begin{example}
Which of the following are valid overloadings?

\noindent A:
\begin{code}
public int absValue(int x) {
  if (x < 0) {
    return x * -1;
  }
  return x;
}
public double absValue(int x) {
  if (x < 0) {
    return 0.0 - x;
  }
  return x - 0.0;
}
\end{code}

\noindent B:
\begin{code}
public int absValue(int x) {
  if (x < 0) {
    return x * -1;
  }
  return x;
}
public double absValue(double x) {
  if (x < 0) {
    x = 0.0 - x;
  }
  return x;
}
\end{code}

\noindent C:
\begin{code}
public int absValue(int x) {
  if (x < 0) {
    return x * -1;
  }
  return x;
}
public int absValueX(int x) {
  if (x < 0) {
    return x * -1;
  }
  return x;
}
\end{code}

\noindent \emph{Answer:}
\begin{itemize}
\item A is not a valid overloading because both methods have the same signature.
\item B is a valid overloading because both methods have different signatures.
\item C is valid code, but this is not a valid overloading. It is not overloading at all because the methods have different names.
\end{itemize}
\end{example}

\subsection{Methods Summary}
In summary, every method definition has the following:
\begin{itemize}
\item An access modifier (if not explicitly stated, this will be the \emph{default} modifier)
\item A return type (possibly \ic{void})
\item A name
\item A (possibly empty) list of parameter types and names that can be passed to the method when it is called
\item A body
\item A return statement reached on every control path in the body, where each \ic{return} is followed by an
expression of the method's return type (unless the return type is \ic{void})
\end{itemize}
Each method is identified by its signature, so you may have methods with the same name but different parameter lists.

\section{Abstraction}\label{sec:abstraction}
Let us now return to our problem of finding the maximum scores for each
of the three groups.
The code for each of the three groups, though similar, have different variable names;
however, this is the only way in which they are different.
The process of distilling the similarity among different pieces of code is
the process of \emph{abstraction}.
In this case, we would like to \emph{abstract} away the details of
each copy to achieve a piece of code that describes all of their behavior.

After abstracting away the variable names, the code for each group is such that,
if we use the appropriate variable names for the students in the group
instead of \ic{x}, \ic{y}, and \ic{z}, that the following code snippet
would describe all of the three different operations, with
\ic{max} storing the desired result:
\begin{code}
int max = x;
if (y > max) {
  max = y; 
}
if (z > max) {
  max = z;
}
\end{code}
We can wrap this code snippet up in a method definition
that returns the value we care about:
\begin{code}
public int max3(int x, int y, int z) {
  int max = x;
  if (y > max) {
    max = y;
  }
  if (z > max) {
    max = z;
  }
  return max;
}
\end{code}

If we use this method in place of the redundant code,
our original code that compute the maximum scores for each of the three
groups then becomes as follows:
\begin{code}
int group1Max = max3(group1s1, group1s2, group1s3);
int group2Max = max3(group2s1, group2s2, group2s3);
int group3Max = max3(group3s1, group3s2, group3s3);
\end{code}
This resulting code is much more concise and is an example
of \emph{code reuse}, where the same exact code
(i.e. the code inside the body of \ic{max3}) is being reused
in several places. This code is also perhaps easier to understand
if you know that \ic{max3} simply calculates the maximum
of its three arguments. In the original code, after
going through the code for one of the groups
and realizing that it computes a maximum, you
would have to go through the calculations of the other
groups to make sure that they also compute a maximum.
Here, it is easy to see that the same computation is happening
for each group (but with different inputs).

We might further notice that both of the conditional statements
in the \ic{max3} method
compute very similar things (i.e. the maximum of two numbers),
and might perform further abstraction to achieve the following
method:
\begin{code}
int max2(int x, int y) {
  int max = x;
  if (y > max) {
    max = y; 
  }
}
\end{code}

We can then adjust our \ic{max3} method to call this one:
\begin{code}
int max3(int x, int y, int z) {
  return max2(max2(x, y), z);
}
\end{code}

Note that we do not need to change our code
for computing \ic{group1Max}, \ic{group2Max},
or \ic{group3Max} in order to use the method
\ic{max2}. The changes to \ic{max3} are sufficient.

\section{Modularity (Optional)}\label{sec:modularity}
Let us again consider the case in which our calculation of the maximum
score is incorrect because we used \ic{<} instead of \ic{>} in
the last comparison. Let us assume that we are still using the
version of \ic{max3} that does not call \ic{max2}, but
we mistakenly have the comparison \ic{z < max} instead of
\ic{z > max} in our method:
\begin{code}
public int max3(int x, int y, int z) {
  int max = x;
  if (y > max) {
    max = y;
  }
  if (z < max) {
    max = z;
  }
  return max;
}
\end{code}

In order to fix, this we simply need to change \ic{z < max}
to \ic{z > max} once. The code that calculates
the values of \ic{group1Max}, \ic{group2Max},
and \ic{group3Max} by calling \ic{max3} is fixed by this
one change because for all the groups, we call the
same method. Here we have achieved a \emph{separation of concerns}
in the two parts of our code:
\begin{itemize}
\item We have one part of our code (the \ic{max3} method)
that is concerned with finding the maximum of three \emph{arbitrary}
numbers, but it is not concerned with \emph{which} numbers
specifically for which it is finding the maximum.
\item We have another part that
is concerned with computing the maximum scores for each group,
\emph{given that} we have some other code that
can calculate the maximum of three numbers, but it is
not concerned with \emph{how} this maximum is found,
as long as it is done correctly
\end{itemize}

This separation of concerns is referred to as \emph{modularity},
and we can regard the code inside of different methods as
being in different \emph{modules}.
We have already seen that modularity can allow us to do
things like fix bugs in one part of our code without having
to touch other parts of our code.

Modularity also lets us do other things, like naturally
divide up labor. If you wanted to split up the work
of writing a program that finds the maximum of each
group's three scores with a friend,
you might volunteer to write the code that deals with
finding the maximum scores for each group provided that
your friend writes code that finds the maximum of three numbers
contained in a method with signature \ic{max3(int, int int)} that returns
an \ic{int} that is the maximum of its three \ic{int} arguments.

Given that you know your friend will write such a \ic{max3}
method, you can, without seeing your friend's code,
write the following code (that we have already seen):
\begin{code}
int group1Max = max3(group1s1, group1s2, group1s3);
int group2Max = max3(group2s1, group2s2, group2s3);
int group3Max = max3(group3s1, group3s2, group3s3);
\end{code}

\noindent Once your friend finishes writing \ic{max3}, then your code
should work as expected. 

An advantage of modularity is that it is not necessary
to know the details of how the modules that you use
are implemented nor used.
In the above example, you do not need to know how exactly your
friend implements \ic{max3} (maybe it calls \ic{max2}
and maybe it does not) and your friend does not need to
know how you use \ic{max3} (maybe you also use it to
find the maximum of the three maximum scores and maybe
you do not).

\begin{example}
What would you add to the following code to calculate
the maximum of \ic{group1Max}, \ic{group2Max},
and \ic{group3Max} and store it in \ic{allGroupMax}?

\begin{code}
int group1Max = max3(group1s1, group1s2, group1s3);
int group2Max = max3(group2s1, group2s2, group2s3);
int group3Max = max3(group3s1, group3s2, group3s3);
\end{code}

\noindent \emph{Answer:}
\begin{code}
int group1Max = max3(group1s1, group1s2, group1s3);
int group2Max = max3(group2s1, group2s2, group2s3);
int group3Max = max3(group3s1, group3s2, group3s3);
// added code below:
int allGroupMax = max3(group1Max, group2Max, group3Max);
\end{code}
\end{example}

\begin{example}
Consider the case where not only do you
want to compute the maximum scores for each group
for Group 1, Group 2, and Group 3, but you
also want to do the same for Group 4 and Group 5.
Unfortunately, while Group 4 and Group 5 also have
three scores per group (\ic{group4s1}, \ic{group4s2},
\ic{group4s3}, \ic{group5s1}, \ic{group5s2},
and \ic{group5s3}), these scores are all \ic{double}s
rather than \ic{int}s.

You want write the following code:
\begin{code}
int group1Max = max3(group1s1, group1s2, group1s3);
int group2Max = max3(group2s1, group2s2, group2s3);
int group3Max = max3(group3s1, group3s2, group3s3);
double group4Max = max3(group4s1, group4s2, group4s3);
double group5Max = max3(group5s1, group5s2, group5s3);
\end{code}

What code do you ask your friend to write
so that your code works?

\noindent \emph{Answer:} You can ask your friend to write
an overloaded \ic{max3}: one overloading (the one
we have seen so far) should return an
\ic{int} and have method signature \ic{max3(int, int, int)}
and the other should return a \ic{double} and have
method signature \ic{max3(double, double, double)}.
Both versions of \ic{max3} should return the maximum value
of their three arguments.
\end{example}

\begin{example}
Give the signatures of methods that you would need to write
to make the following code work:
\begin{code}
int computeAverages(int a, int b, int c, int d, int e, int f, int g, int h) {
  int avg0 = sum(a, b) / 2;
  int avg1 = sum(c, d, e) / 3;
  int avg2 = sum(f, g, h) / 3;
  int avgAvg = sum(avg0, avg1, avg2) / 3;
  return avgAvg;
}
\end{code}

\noindent \emph{Answer:}
\ic{sum(int, int)} and \ic{sum(int, int, int)}
\end{example}

\begin{example}
This is the same code as in the previous example, but
someone made a mistake and divided by 4 instead of 3
when taking the average of three numbers:
\begin{code}
int computeAverages(int a, int b, int c, int d, int e, int f, int g, int h) {
  int avg0 = sum(a, b) / 2;
  int avg1 = sum(c, d, e) / 4;
  int avg2 = sum(f, g, h) / 4;
  int avgAvg = sum(avg0, avg1, avg2) / 4;
  return avgAvg;
}
\end{code}
Improve the original code by using abstraction.

\noindent \emph{Answer:}
\begin{code}
int avg(int a, int b) {
  return sum(a, b) / 2;
}
int avg(int a, int b, int c) {
  return sum(a, b, c) / 3;
}
int computeAverages(int a, int b, int c, int d, int e, int f, int g, int h) {
  int avg0 = avg(a, b);
  int avg1 = avg(c, d, e);
  int avg2 = avg(f, g, h);
  int avgAvg = sum(avg0, avg1, avg2) / 4;
  return avgAvg;
}
\end{code}

\end{example}
