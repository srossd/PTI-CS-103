\chapter{Loop Structures}

Sometimes, it is useful to repeat an action multiple times. Maybe you want to
ask a user for their home address, if they do not provide a valid address, you
ask them for an address again. Repeating an action a number of times in this way
is a commonly used technique in programming. Structures that repeat actions are
called \emph{loop structures}.

In this chapter, we will learn a number of useful loop structures, like
\ic{while} and \ic{for} loops. We describe when to use a \ic{while} loop, and
when to use a \ic{for} loop. The chapter ends with a list of common mistakes to
avoid when writing loops.

\section{A \ic{while} loop}

\begin{definition}
A \emph{while loop} is a \emph{loop structure} that consists of the reserved
keyword \ic{while}, followed by a boolean expression enclosed in parentheses,
followed by statements typically enclosed in curly braces. These statements is
executed as long as the boolean expression remains true. If the expression
evaluates to false, the loop terminates, and the program continues.
\end{definition}

A \ic{while} loop allows us to write programs execute statements
(called the \emph{loop body}), based on a boolean expression (called the \emph{condition}). Below is an example of a simple \ic{while} loop:

\begin{code}
while (count < 20) {
  count = count+1;
  System.out.println("Count less than 20");
}
System.out.println("Count greater than, or equal to, 20");
\end{code}

\noindent The condition in this example is \ic{count > 20}. It is a boolean
expression that evaluates to either true or false. That is, \ic{count} is either
greater than \ic{20} or not. If it is, count is increased by one, and ``Count
less than 20'' is printed. Then, the program tests whether the new count is less
than 20. If it is, the count is again increased by one, and ``Count less than
20'' is printed again. Once count is greater than, or equal to, 20, ``Count
greater than, or equal to, 20'' is printed.

\begin{example}
What does the following piece of code print if \ic{x} is 98? What about if \ic{x} is 200? What about if \ic{x} is 100?

\begin{code}
while (x < 100) {
    x = x+1
    System.out.println("Not big enough...");
} 
System.out.println("Big enough!");
\end{code}

\emph{Answer}: If \ic{x} is 98, the first boolean expression will evaluate to
true, so the program will print ``Not big enough...'' Then the program will
loop, and check the condition again.  Again, it will evaluate to true, and print
``Not big enough...'' In the next iteration, x will be 100, and the conditional
will evaluate to false, and will print out ``Big enough.''

Using the same reasoning to trace the execution with \ic{x} is 200, the
following statements will be printed: ``Big enough!''.

When \ic{x} is 100, the condition is just barely big enough, so ``Big enough''
will be printed.
\end{example}

\begin{example}
How would you fill in the boolean expression to print every number downwards by 1 from the initial value of x, until x is less than 0?.

\begin{code}
while (/* Insert boolean expression here */) {
    x = x-1;
} 
\end{code}

\emph{Answer}: Insert the boolean expression x $>=$ 0. Note that it must be $>=$ rather than just $>$ since we still want to print if x is exactly 0 (only stop if x is less than 0). 
\end{example}

\begin{example}
How would you fill in the loop body to print all even numbers from 0 to 10.

\begin{code}
int x = 0;
while (x <= 10) {
    /* Insert loop body here */
} 
\end{code}

\emph{Answer}: Insert the statements:
\begin{code}
  System.out.println(x);
  x = x+2;
\end{code}
\end{example}

\section{The \ic{for} loop}
While a \ic{while} loop can express all possible loops, many while loops follow
the same pattern. Before the loop, you initialize a variable; the loop condition
involves that variable; and within the loop body, you then update that variable
in the same way. When loops involves this pattern, programmers typically write a
\ic{for} loop. Below is a \ic{for} loop that prints all the numbers from 0 to 9.

\begin{code}
for (int i = 0; i < 10; i++)
{
    System.out.println(i);
}
\end{code}

A \ic{for} loop has 4 parts, initialization, condition, change, and body. In the
above \ic{for} loop, \ic{int i = 0} is the initialization. It initializes a new
variable, \ic{i}, to be zero. Next is the condition, \ic{i < 10}. This is the
condition that describes when the loop ends. Then comes the change, \ic{i++}.
After the loop body is run, the change is run as well.  Finally is the loop
body, which in this case merely prints out \ic{i}.

Note that this loop could have been written as a while loop, shown below:
\begin{code}
int i = 0;
while (i < 10)  
{
    System.out.println(i);
    i++
}
\end{code}
%
However, programmers use a \ic{for} loop, they know that the loop structure has
the structure of requiring initialization, having a loop condition (typically
refer to the initialized variable), and having a change on each loop
iteration (typically changing the initialized variable), and having a loop body
(that typically does not change the initialized variable).

\begin{example}
What does the following piece of code print?

\begin{code}
for (int i = 10; i > 0; i--)
{
    System.out.println(i);
}
\end{code}

\emph{Answer}: 
10
9
8
7
6
5
4
3
2
1
\end{example}

\begin{example}
What does the following piece of code print?

\begin{code}
for (int j = 1; j<100; j = j*2)
{
    System.out.println(j);
}
\end{code}

\emph{Answer}: 
1
2
4
8
16
32
64
\end{example}

\begin{example}
What would you put in the initialization, condition, and change to print all
even numbers from 0 to 20 with a new line in between them?

\begin{code}
for (?; ?; ?)
{
    System.out.println(i);
}
\end{code}

\emph{Answer}:
\begin{code}
for (int i = 0; i < 20; i = i+2)
{
    System.out.println(i);
}
\end{code}
\end{example}

\section{Nested loops}
It is possible to combine loop statements in many interesting ways. Consider the following example:

\begin{code}
for (int i = 0; i < 100; i++)
{
    System.out.println("Go up to: " + i);
    for (int j = 0; j < i; j++)
    {
        System.out.println(j);
    }
}
\end{code}

\noindent For each number from 0 to 99, this code will print all the numbers up
to (and not including) it.  For example, it will start by printing \ic{0}. Then it will print \ic{0}
and \ic{1}.  Then it will print \ic{0}, \ic{1}, and \ic{2}.  This will continue
until it prints all the numbers from \ic{0} to \ic{98}.

\section{Curly braces}
So far, we have been using curly braces to enclose each of our loop bodies. These curly braces can be omitted if there is only a single statement. For example, in the snippet of code below, the first set of curly braces can be omitted, while the second cannot.

\begin{code}
int i = 0;
while (i < 10)
{
    i++;
}
i = 0;
while (i < 10)
{
    System.out.println(i);
    i++;
}
\end{code}

\noindent If all the curly braces were left out (as in the code below), the
program would complete the first loop, printing nothing. Then it would begin the
second loop, and print 0 forever..

\begin{code}
int i = 0;
while (i < 10)
    i++;
i = 0;
while (i < 10)
    System.out.println(i);
    i++;
// ^^^ Careful! This is not part of the while loop!
\end{code}

Here it is important to note that whitespace and indentation are ignored by Java. Indentation has no effect on the behavior of a program. Proper indentation is extremely important for human readability. When used incorrectly, however, misleading indentation can result in unexpected behavior.

\exercisesection

\begin{exercise}
What output is produced by the following code fragment?

\begin{code}
int i = 0;
int j = 0;
while (i < 3)
{
  j = j+i;
  System.out.println(j);
  i++;
}
\end{code}
\end{exercise}

\begin{exercise}
Convert the following loop into a \ic{for} loop.

\begin{code}
int num = 100;
while (num > 0)
{
    System.out.println(num);
    num = num / 2;
}
\end{code}
\end{exercise}

\begin{exercise}
  Write code that prints out ``hi'' 10 times, once without using loops, and once
  using loops.
\end{exercise}

\referencessection